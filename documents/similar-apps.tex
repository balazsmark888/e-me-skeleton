\documentclass[12pt]{article}
\usepackage[utf8]{inputenc}
\title{(Not so)Similar applications}
\author{Mark Balazs}
\date{2021–03–20}
\begin{document}
\maketitle
    \section{Applications}
        \begin{enumerate}
        \item Google Docs
            \begin{itemize}
                \item create and edit documents
                \item sync between multiple devices
                \item view PDF docs/presentations
                \item upload and manage files
            \end{itemize}
        \item Documents to Go
            \begin{itemize}
                \item edit/view/create word, excel, PowerPoint docs
                \item supports password protection
                \item Google Docs support
                \item bi-directional sync
            \end{itemize}
        \item SecureSafe
            \begin{itemize}
                \item secure file and data storage
                \item double encryption
                \item secure AES-256 and RSA-2048 encryption
                \item https
                \item MFA with SMS
                \item send files up to 2GB to recipients
            \end{itemize}
        \item Quick Office Pro
            \begin{itemize}
                \item create/edit/share Microsoft Office files
                \item offline file access
            \end{itemize}
        \end{enumerate}
    \section{Contrast}
        \subsection{Similarities}
            \begin{itemize}
                \item Similarly to the \textbf{SecureSafe} app, E-me uses AES-256 symmetric encryption standard to securely store and transfer documents.
                \item E-me allows users to upload their PDF documents.
                \item Users have quick and secure access to their data and files.
            \end{itemize}
        \subsection{Differences}
            \begin{itemize}
                \item E-me only supports PDF documents.
                \item E-me uses End-to-End Encryption over HTTPS to communicate with the clients.
                \item Users are able to \textbf{generate} their PDF docs using predefined templates filled out with their personal data.
                \item All PDF documents \textbf{(generated or uploaded)} will be verified for authenticity by the system administrators (later government) and will receive a digital signature to mark their authenticity.
                \item Authorities can request access to users' documents in order to verify their identity or other personal information (this access is temporary). 
            \end{itemize}

\end{document}