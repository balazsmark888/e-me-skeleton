%!TEX root = thesis.tex
%%%%%%%%%%%%%%%%%%%%%%%%%%%%%%%%%%%%%%%%%%%%%%%%%%%%%%%%%%%%%%%%%%%%%%%
\chapter{Introduction}\label{ch:INTRO}
%%%%%%%%%%%%%%%%%%%%%%%%%%%%%%%%%%%%%%%%%%%%%%%%%%%%%%%%%%%%%%%%%%%%%%%

%%%%%%%%%%%%%%%%%%%%%%%%%%%%%%%%%%%%%%%%%%%%%%%%%%%%%%%%%%%%%%%%%%%%%%%
\section{About E-me}\label{sec:INTRO:about}

E-me is a platform that serves the purpose of handling various PDF documents and user-related personal information in the form of an Android application.

The main focus of E-me is to provide quick and easy access to documents for its users that otherwise would require a physical paper format to be presented.
It aims to replace the traditional physical documents while maintaining validity and data integrity.
By means of this replacement, E-me not only creates a collection of a user's documents that can be accessed any time using a mobile device, but also
ensures the security of these documents in order to protect its users' privacy.

The application targets people from all walks of life, since every person owns legal documents that often times are misplaced or even lost.
Nowadays, the everyday life of an average person is becoming more and more filled with either work or family-related tasks and responsibilities.
This increasing number of activities a person has to carry out in a day takes a great amount of effort and more importantly it consumes a substantial part
of their time.
When it comes to adding new obligations to this heap of duties, that may even depend on various schemes of public institutions and/or authorities,
one might have to leave out or even cancel otherwise important daily projects.
Acquiring new legal documents takes a great amount of time and patience, since it requires the presence of the owner of the corresponding document, 
not to mention the transportation and queue time.

E-me intends to tackle this problem by providing its users the ability to request documents based on predefined templates (ex. affidavit).
This way a user is able to avoid the cumbersome process of preparing and verifying paperwork.
In addition to saving time for a person, the application processes the personal information provided by the user in order to 
automatically complete the necessary fields of the documents that are being requested.
This ensures that the documents created by the application are filled out using the correct information, not giving room for manual/human errors.

\section{Similar applications in the field}\label{sec:INTRO:sa}

As a document-handling mobile application, E-me has its fair share of competitors in the field.
Although there is a staggering number of applications for document management, the feature-set and characteristics of E-me make it 
unique in multiple ways.

\subsubsection{Google Docs}

Being one of the most popular document-management application, Google Docs provides a wide variety of features and flexibility in terms of 
collaboration and document authoring.
It can handle multiple types of documents such as PDF, Excel, PowerPoint, Word and many more.
One of its key features is document creation and editing.
Multiple users are able to edit documents at the same time while seeing each others' changes instantaneously.
Google Docs also supports offline editing on multiple platforms and is able to synchronize these documents once the devices come online.

Being a Google product, Google Docs is well-integrated into other commonly used applications (ex. Google Drive) which makes 
it an effortless collaboration platform for easy cloud-based document access.
It has a paid version called Google Workspace which provides additional security features for data protection and privacy.

\subsubsection{Documents to Go}

Documents to Go is one of the most popular document viewing apps in the market.
There is a free as well as paid version available for users.
The free version supports viewing file formats including Excel, Word and PowerPoint, however the editing of these files requires the purchase
of the full version of the application.

The full version includes other practical features such as creating and editing PDF documents, password-protecting Word and Excel files
and it also has Google Docs support, meaning a user can view and edit their files from Google Docs directly in Documents to Go.
The application is available on Android as well as on Windows, providing an easy way to transfer documents across multiple devices.

\subsubsection{SecureSafe}

SecureSafe is an online storage solution.
This cloud safe simplifies online file sharing and protects documents and/or passwords, offering a high level of security by means of encryption.
The platform relies heavily on cryptography and offers a zero-knowledge protocol for its users.
It uses double-ecryption combined with triple redundant data storage in order to prevent security breaches and/or potential data loss.

Being a cloud-hosted application, it enables encrypted file synchronization across multiple devices of a user, but also provides
end-to-end encrypted file and pin sharing up to 2 GB in size to any recipients.
It also has an integrated password manager and generator for highly secure pins for the online accounts of its users.

\subsection{Comparison}

\subsubsection{Similarities}

Since E-me itself is a document-handling application, much like the above mentioned applications, it also provides its users the possibility to view,
create, edit or even delete their documents within the application.
Similarly to \emph{SafeSecure}, the application uses encryption to protect the privacy of its users and uses secure HTTPS-based communication
to ensure data protection.

Much like the other applications, E-me has a built-in document sharing mechanism which allows for efficient and secure data transfer between devices of
 the same or even different users. 


\subsubsection{Differences}

When compared to other file-handling applications, it is important to note that E-me only supports PDF documents.
This limitation of file types derives from the fact that, in its core, the application focuses on generating and managing legal documents which
are best represented as PDF files when it comes to digitalisation.

Another unique aspect of E-me is allowing the user to provide their personal information.
This will be securely stored on the server and processed whenever a user requires a new document.
Documents generated by the application are based on predefined templates that represent legal papers.
These usually include form fields that are to be completed by the user, however E-me specializes in generating pre-filled documents that require 
much less or even no further modifications from the user.

Finally, sharing a document via E-me is not permanent.
The recepient of the file only has a one-time token when it comes to accessing the corresponding document in a form of a QR code.
This code is generated by the owner of the document and it can be interpreted via the built-in scanner of the application in order to 
retrieve the specified document for a short amount of time. 

\section{Summary}\label{sec:INTRO:sum}

In the following chapters the application will be presented from various perspectives.
The second chapter describes the project from a user point of view.
It describes the visuals of the application using illustrations and provides a detailed explanation of the various features and use cases
of the app.

In the third and larges chapter are discussed the technical details of the application.
It presents the architecture of the project using diagrams, as well as the technologies that were utilized during development.
After that, it describes the implementation of some of the major layers and use cases with examples from the actual code.
The last part of the chapter discusses the security features of the application, the cryptographic algorithms and methods that were used
to ensure user privacy within the app.

The last chapter contains measurements about some of the cryptographic functions that are integrated in the application, furthermore
it provides information about the system requirements for both the mobile and the server parts of the project.
The second half of the chapter presents some of the major difficulties and obstacles that occured during the development process, as well as 
possible feature upgrades and expansions for the application.